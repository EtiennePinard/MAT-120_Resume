% The asterisk is for the unumbered section
\section*{Chapitre 1: Nombres Complexes}

\subsubsection*{Représentation des nombres complexes}
% Please note that I used the three ways to enter inline math mode in this list. This is just to use them as I am still a beginner at Latex
\begin{itemize}[itemsep = 0.5em]
    \item[] Forme algégrique (cartésienne): \begin{math}z = x + iy \quad x,y \in \mathbb{C} \end{math}
    \item[] Forme polaire: $z = r(\cos(\theta)+i\sin(\theta)) \quad r, \theta \in \mathbb{R}$
    \item[] Forme exponentielle: \(z = re^{i\theta} \quad r, \theta \in \mathbb{R} \)
  \end{itemize}


\subsubsection*{Changement de forme}
\begin{itemize}
    \item[] Soit $z = x + iy = r(\cos(\theta)+i\sin(\theta)) = re^{i\theta}$
    \item[] Alors on a,
    \begin{align*}
    % The &= is to align by the equal sign
    r &= |z| = \sqrt{x^2 + y^2} \\
    \theta &= Arg(z) =
    \begin{cases} 
        \arctan(\frac{y}{x}) & x > 0 \\
        \arctan(\frac{y}{x}) - \pi & x < 0 \\
        \pm\frac{\pi}{2} & x = 0, \textit{même signe que y}
    \end{cases}
    \end{align*}
    \item[] où $r$ est le module et $\theta$ est l'argument de $z$ 
\end{itemize}


\subsubsection*{Opérations propres aux nombres complexes}
\begin{itemize}
    \item[] Soit $z = x + iy = r(\cos(\theta)+i\sin(\theta)) = re^{i\theta}$
    \begin{itemize}[itemsep = 0.5em]
        \item[] Conjugée: $z^{*} = x - iy$
        \item[] Partie réel: $Re(z) = \frac{z + z^{*}}{2} = x, \quad Re(z) \in \mathbb{R}$
        \item[] Partie imaginaire: $Im(z) = \frac{z - z^{*}}{2} = y, \quad Im(z) \in \mathbb{R}$
        \item[] Argument (non-unique): $\arg(z) = Arg(z) + 2\pi k, \quad k \in \mathbb{Z}$
      \end{itemize}
\end{itemize}


\paragraph*{Propriétés des opérations}
% This group syntax is to add a bit more space after a linebreak for the properties list
\begingroup
\addtolength{\jot}{0.5em}
\begin{align*}
    \left(z_1 + z_2\right)^* &= z_1^* + z_2^* \\
    \left(z_1z_2\right)^* &= z_1^{*} z_2^* \\
    \left( \frac{z_1}{z_2}\right)^{*} &= \frac{z_1^*}{z_2^*} \\
    \left(z^*\right)^* &= z \\
    zz^{*} &= |z|^2 \\
    |z_1z_2| &= |z_1||z_2|\\
    \left| \frac{z_1}{z_2} \right| &= \frac{|z_1|}{|z_2|} \\
    \left|z_1 + z_2\right| &\leq |z_1| + |z_2| \quad (\textit{Inégalité du triangle}) \\
    \arg(z_1z_2) &= \arg(z_1) + \arg(z_2) \\
    \arg\left( \frac{z_1}{z_2}\right) &= \arg(z_1) - \arg(z_2)
\end{align*}
\endgroup


\subsubsection*{Racines entières}
\begin{itemize}
    \item[] Soit $z = re^{i\theta},\ n \in \mathbb{N}$.
    \item[] Alors on a,
\end{itemize}
    \begin{equation*}
        \sqrt[n]{z} = \sqrt[n]{r} \left( e^{i \left( \frac{\theta + 2\pi k}{n} \right) } \right), \ k = 0, \dots, n - 1
    \end{equation*} 
\begin{itemize}
  \item[] On peut aussi définir les racines de manières récursives.
  \item[] Si $w$ est une racine de $z$, alors on a
\end{itemize}
    \begin{equation*}
        w_k = \begin{cases} 
            w_{k - 1}e^{ \frac{2\pi i}{n} } & k > 0 \\[0.5em]
            \sqrt[n]{r} \left( e^{i \frac{\theta}{n} } \right) & k = 0
        \end{cases}
    \end{equation*} 
\begin{itemize}
   \item[] On remarque que deux racines entières consécutives sont séparées par un angle de $\frac{2\pi}{n}$    
\end{itemize}


\subsubsection*{Exponentielle  et logarithme}
\begin{itemize}
    \item[] Soit $z = x + iy$. 
    \item[] Alors on a,
\end{itemize}
\begin{equation*}
    e^z = e^{x + iy} = e^x e^{iy} = e^x (cos(y) + isin(y)) 
\end{equation*}
\begin{itemize}
    \item[] De plus,
\end{itemize}
\begin{equation*}
    \ln(z) = \log(z) = \ln(|z|e^{i(\theta + 2\pi k)}) = \ln|z| + i(\theta + 2\pi k), \quad k \in \mathbb{Z}
\end{equation*}
\begin{itemize}
    \item[] où $(\theta + 2\pi k) = \arg(z)$ 
    \item[] Donc, $\ln(z) = \ln|z| + i\arg(z)$
\end{itemize}


\subsubsection*{Puissance Complexe}
\begin{itemize} 
    \item[] Soit $z \in \mathbb{C}$ et $w= x + iy$
    \item[] Alors on a,
    \item[] \begin{align*}
        z^w &= \left( e^{\ln(z)}\right)^{w} = e^{\ln(z)w} \\ 
        &= e^{\left(\ln|z| + i\arg(z)\right)\left(x + iy\right)} \\
        &= e^{ x\ln|z| - y\arg(z) + i\left( x\arg(z) + y\ln|z|\right) } \\
        &= e^{x\ln|z| - y\arg(z)}e^{i\left( x\arg(z) + y\ln|z| \right)} \\
        &= |z|^{x} e^{-y\left( Arg(z) + 2\pi k \right)} e^{ i \left( y\ln|z| + xArg(z) \right) }, \; k \in \mathbb{Z}
    \end{align*}
    \item[] Alors $z^w$ prend une infinité de valeur.
    \item[] Si $z \in \mathbb{R},\; z > 0$ et $w \in \mathbb{Q}'$, l'equation devient $z^w = z^w e^{i 2 \pi k w}$ 
    \item[] Cette equation a une infinité de valeurs puisque $kw \not\in \mathbb{Z} \implies e^{i 2 \pi k w} \neq 1, \forall z, w$
    \item[] Dans ce cas la convention est de prendre $k = 0$ pour que $z^w \in \mathbb{R}$
\end{itemize}   


\subsubsection*{Théorème fondamental de l'algèbre}
\begin{itemize}
    \item[] Soit $p(x) = a_n x^n + a_{n-1} x^{n-1} + \dots + a_1 x + a_0, \; a_j \in \mathbb{C}, \; a_n \neq 0, \; n \in \mathbb{N}$
    \item[] Alors on peut écrire $p(x)$ en terme de ses racines, soit
    \item[] \begin{equation*}
        p(x) = a_n(x - x_1)^{k_1} (x - x_2)^{k_2} \dots (x - x_m)^{k_m}
    \end{equation*}
    \item[] où $x_1, x_2, \dots ,x_m$ sont les racines de $p(x)$ et $k_j$ est la multiplicité de la racine $x_j$
    \item[] Le théorème fondamental de l'algèbre nous dit que $k_1 + k_2 + \dots + k_m = n$
    \item[] On peut donc dire que $p(x)$ à exactement $n$ racines complexes en comptant les multiplicités.
\end{itemize}